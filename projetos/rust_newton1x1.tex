\begin{lstlisting}[caption=Método de Newton 1 variavel]
pub fn newton1x1<F>(funcao_derivada: &F, x: f64) -> (usize, f64)
where
    F: Fn(f64) -> f64,
{
    let mut entrada_atual = x;
    let maximo_iteracoes = 100;

    for iteracao_atual in 1..=maximo_iteracoes {
        let diferenca: f64 =
            funcao_derivada(entrada_atual.clone())
            /
            derive1x1(&funcao_derivada, &entrada_atual);

        println!("diferenca: {}", diferenca);
        entrada_atual -= diferenca;

        if diferenca.abs() < 0.0000001 {
            return (iteracao_atual, entrada_atual);
        }
    }

    return (maximo_iteracoes, entrada_atual);
}
\end{lstlisting}
