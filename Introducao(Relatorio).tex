%%%%%%%%%%%%%%%%%%%%%%%%%%%%%%%%%%%%%%%%%%%%%%%%%
% Geometria Analítica
% Gersonilo Oliveira da Silva 
% Introdução
% %%%%%%%%%%%%%%%%%%%%%%%%%%%%%%%%%%%%%%%%%%%%%%%%%

\pagenumbering{arabic} \setcounter{page}{6} \thispagestyle{empty}
\addcontentsline{toc}{section}{{\bf Introdução}}
\chapter*{\textbf{\LARGE{Introdução}}}


\vspace{2cm}

\begin{center}
\textit{Para Tales, a questão primordial não era o que sabemos,\\ mas como o sabemos.}
\end{center}

\begin{flushright}
\textsl{Aristóteles}
\end{flushright}

\begin{center}
\textit{Evita o que o pertuba a mente e o que a alma esmaga,\\ Aprimora a razão, esmera os valores teus;\\ E transpondo, enfim, a prefulgante plaga\\
Tu, entre os imortais, serás também um deus.}
\end{center}

\begin{flushright}
\textsl{Pitágoras}
\end{flushright}

\begin{center}
\textit{Não posso me convencer de que, quando se soma uma a um, o um a que foi\\
feita a adição se transforma em dois, ou que duas unidades somadas farão\\ dois em consequência da adição. Não posso entender como quando\\
separadas cada uma era uma e não dois e agora, quando reunidas, a\\ simples justaposição ou encontro delas seja causa de se tornarem dois.}
\end{center}

\begin{flushright}
\textsl{Diálogo de Platão}
\end{flushright}

\begin{center}
\textit{É, pois, sem razão, que os geômetras são acusados de ensinarem apenas\\ quimeras e de não terem na sua ciência nada de bom e de belo. Eu, pelo\\ contrário, sustento que eles, sem disso fazerem ostentação, ensinam coisas\\ que são, ao mesmo tempo, muito boas e muito belas. Pois toda bondade e\\ toda beleza não resulta, forçosamente, da ordem e da proporção? Ora, de\\ que coisas se ocupa o geômetra, senão da ordem e da porporção?}
\end{center}

\begin{flushright}
\textsl{Aristóteles - Tratado de Filosofia}
\end{flushright}

\begin{center}
\textit{Ptolomeu uma vez perguntou se havia um caminho mais curto para a\\ geometria que o estudo de Os elementos e Euclides lhe respondeu que não\\ havia estrada real para a geometria.}
\end{center}

\begin{flushright}
Proclus Diadocus
\end{flushright}


%\par Aqui fazemos uma análise geométrica dos instrumentos matemáticos modernos a fim de expor em linha reta o que se faz necessário ao conhecimento de um Bacharel em Ciência da Computação. Em nosso primeiro capítulo discorremos sobre os efeitos históricos do desenvolvimento da Geometria partindo de sua origem, sendo esta considerada literariamente, em Os elementos de Euclides, partindo para o descorrer da origem das coordenadas na geometria, feita por Descartes. Depois partimos para a descrição sucinta da teoria das matrizes no segundo capítulo. Feita muito respaldada na ementa do curso. Seguindo com a geometria propriamente dita. Considerando os espaços $\mathbb{R}^2$ e $\mathbb{R}^{3}$ seus elementos e suas propriedades. Depois vemos as curvas provindas das seções cônicas, e por conseguinte as superfíceis quadrádicas. Findando deste modo o propósito deste compêndio.
