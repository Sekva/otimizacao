%%%%%%%%%%%%%%%%%%%%%%%%%%%%%%%%%%%%%%%%%%%%%%%%%
% Relatório Final - Projeto de Pesquisa
% Métodos de Otimização
% Baltz & Machado
% Capítulo 2
%%%%%%%%%%%%%%%%%%%%%%%%%%%%%%%%%%%%%%%%%%%%%%%%%


\chapter{\Large{Métodos Clássicos de Otimização}}\label{chp:2}


\section{{O Método de Newton}}

\hspace{0.8cm}

O Método de Newton nos diz que uma sequência \(\{x_k\}\) converge para o mínimo
de \(f(x)\):

\begin{equation}
    x_{k+1} = x_{k} - \frac {f'(x_{k})}{f''(x_{k})}
\end{equation}

O que este método faz de fato é encontrar zeros de uma função \(g(x)\) a partir
da mesma sequência \(\{x_k\}\), afim de que \(x_k\) converga para uma entrada
\(x^*\) de \(g(x)\), tal que a mesma se anule:

\begin{equation}
    x_{k+1} = x_{k} - \frac {g(x_{k})}{g'(x_{k})}
\end{equation}

\begin{equation}
    g(x^*) = 0
\end{equation}


Com isso temos um método que minimiza uma função que pode ser derivada duas
vezes (caso não possa, aproximações de suas primeiras e segundas derivadas
podem ser boas o suficiente) a partir de um valor de entrada. O método é
simples, entrega muitas vezes ótimos locais próximos ao ponto inicial, mas tem
seu destaque pode ser curto e facilmente computável.

Problemas de maximização podem ser vistos sob o seguinte olhar:

\begin{equation}
    max(f(x)) = min(-1 * f(x))
\end{equation}

Que com isto podem ser otimizados pelo Método de Newton também.

O movimento de \(x_k\) dentro da sequência, é determinado pela relação das
quantidades e propriedades que tanto a primeira quanto a segunda derivada
oferecem. As quantidades determinam a velocidade do motivento, e os sinais
indicam a direção do movimento. De certa forma podemos ver esse movimento da
sequência \(\{x_k\}\) como instantes do movimento de uma bola numa ladeira, que
no começo de sua descida é acelerada, e, conforme chega ao plano no fim da
ladeira, começa a reduzir sua velocidade, até supostamente chegar no ponto mais
baixo.


\section{{Outros Métodos}}

\hspace{0.8cm}

\section{{Programando os Métodos}}

\hspace{0.8cm}

\textbf{Método de Newton}

A forma mais simples e mais util de implementar o metodo de Newton é na forma
de busca das raizes, que uma vez  implementada, só precisamos por como entrada
a priemira e a segunda derivada da função que desejamos minimizar, já que o
método não precisa saber qual a função de fato. A seguir temos a implementação
na liguagem de programação Rust:

\begin{lstlisting}[caption=Método de Newton 1 variavel]
pub fn newton1x1<F>(funcao_derivada: &F, x: f64) -> (usize, f64)
where
    F: Fn(f64) -> f64,
{
    let mut entrada_atual = x;
    let maximo_iteracoes = 100;

    for iteracao_atual in 1..=maximo_iteracoes {
        let diferenca: f64 =
            funcao_derivada(entrada_atual.clone())
            /
            derive1x1(&funcao_derivada, &entrada_atual);

        println!("diferenca: {}", diferenca);
        entrada_atual -= diferenca;

        if diferenca.abs() < 0.0000001 {
            return (iteracao_atual, entrada_atual);
        }
    }

    return (maximo_iteracoes, entrada_atual);
}
\end{lstlisting}


Os parâmetros da função são:

    \begin{itemize}
            \item Uma função \(f : \mathbb{R} \rightarrow \mathbb{R}\)
            \item Uma entrada x sendo o chute inicial do otimo.
    \end{itemize}


A função derive1x1 recebe como parametro uma função e um ponto, tendo como
retorno a derivada da função entregue, no ponto especificado. Restingindo-se
à funções do tipo \(f : \mathbb{R} \rightarrow \mathbb{R}\), donde deve ser
escrita pelo úsuario como bem entender.



\textcolor[rgb]{1,0,0}{\section{{O Método de Newton para Várias Variáveis}}}
