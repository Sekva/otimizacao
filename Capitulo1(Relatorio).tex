%%%%%%%%%%%%%%%%%%%%%%%%%%%%%%%%%%%%%%%%%%%%%%%%%
% Relatório Final - Projeto de Pesquisa
% Métodos de Otimização
% Baltz & Machado
% Capítulo 1
%%%%%%%%%%%%%%%%%%%%%%%%%%%%%%%%%%%%%%%%%%%%%%%%%


\chapter{\Large{Métodos Matemáticos de Otimização}}\label{chp:1}


\section{{O Conceito de Otimização}}

\hspace{0.8cm}
Diz-se otimização, o processo que tem como objetivo encontrar condições que
minimizam ou maximizam algo (seja dinheiro, tempo, quantidade, etc). Sendo este,
muitas vezes um trabalho árduo, custoso.

Dessa maneira, na matemática, este processo é amplamente utilizado quando
busca-se valores pertencentes ao conjunto \textit{A} (que pode ter
restrições), com o objetivo de encontrar uma solução ótima, aplicando os valores
de \textit{A} em numa função objetivo predefinida.

Podendo assim, serem representadas da seguinte forma:

	Dada a função
	\begin{equation}
		f : A \rightarrow \mathbb{Z}
	\end{equation}

	\begin{itemize}
		\item Maximização pode ser definida como:
	\end{itemize}

		% TODO: Porque 'que satisfaz' está ficando errado?
		busca pelo elemento \textit{x_0} \in \textit{A}, que satisfaz:

			\begin{equation}
				f(x_0) \geq f(x);
			\end{equation}

		para todo x \in \textit{A}.\\


	\begin{itemize}
		\item Minimização pode ser definida como:
	\end{itemize}

		% TODO: Porque 'que satisfaz' está ficando errado?
		busca pelo elemento \textit{x_0} \in \textit{A}, que satisfaz:

			\begin{equation}
				f(x_0) \leq f(x);
			\end{equation}

		para todo x \in \textit{A}.


\section{{Otimização de Funções à Uma variável real}}

\hspace{0.8cm}

\section{{Programando o Método}}

\hspace{0.8cm}

\textcolor[rgb]{1,0,0}{\section{{Otimização de Funções à Várias Variáveis}}}

\hspace{0.8cm}
