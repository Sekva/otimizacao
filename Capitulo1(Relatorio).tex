%%%%%%%%%%%%%%%%%%%%%%%%%%%%%%%%%%%%%%%%%%%%%%%%%
% Relatório Final - Projeto de Pesquisa
% Métodos de Otimização
% Baltz & Machado
% Capítulo 1
%%%%%%%%%%%%%%%%%%%%%%%%%%%%%%%%%%%%%%%%%%%%%%%%%


\chapter{\Large{Métodos Matemáticos de Otimização}}\label{chp:1}


\section{{O Conceito de Otimização}}

\hspace{0.8cm}
Diz-se otimização, o processo que tem como objetivo encontrar condições que
minimizam ou maximizam algo (seja energia, tempo, dinheiro, etc). Sendo este,
muitas vezes um trabalho árduo, custoso.

Dessa maneira, na matemática, tal processo é amplamente utilizado quando
busca-se valores em conjunto um \textit{A} (que pode ter restrições), com o
objetivo de encontrar uma solução ótima (a melhor resposta para o problema),
aplicando os valores de \textit{A} em uma função objetivo predefinida.

Podendo assim, ser representada da forma como a seguir.

Dada uma função
\begin{equation}
    f : A \subseteq \mathbb{R} \rightarrow \mathbb{R}.
\end{equation}


A maximização pode ser definida como; a busca pelo elemento
\(x^* \in A\), que satisfaz:

\begin{equation}
    f(x^*) \geq f(x),
\end{equation}

para todo \(x \in A\).


A minimização pode ser definida como; a busca pelo elemento \(x^* \in A\),
que satisfaz:

\begin{equation}
    f(x^*) \leq f(x),
\end{equation}

para todo \(x \in A\).

\vspace{\baselineskip}
Com isso, podemos agora entender como esse processo pode ser custoso. Iniciando
com o fato de que existem os pontos máximos e mínimos (pontos
críticos\footnote{Iremos denotar como pontos críticos apenas aqueles onde a função tem derivada e esta se anula.}), locais
e globais, na imagem das funções. Sendo os pontos críticos locais, aqueles que
não são os menores ou maiores valores para a minimização e maximização,
respectivamente. E os pontos globais, aqueles que representam o menor ou maior
valor na imagem das funções, para a minimização e maximização, respectivamente.

Criando, desse modo, uma certa incerteza ao encontrar um valor crítico numa função,
já que, é estritamente difícil saber se o ponto crítico encontrado é local ou
global. Como pode-se perceber na Figura
\ref{grafico_local_global_pontosCriticos}.

\begin{figure}[h]
    \centering
    \includegraphics[width=0.43\textwidth]{src/grafico_local_global_pontosCriticos.png}
    \captionsetup{justification=centering}
    \caption{
      Exemplo de pontos críticos locais e globais indicados no gráfico de uma função.\\
      \tiny Essa imagem foi coletada de \url{https://www.onlinemathlearning.com}.
    }
    \label{grafico_local_global_pontosCriticos}
\end{figure}

Como na grande maioria dos casos não é possível conhecer antecipadamente todo
o espaço gerado pela função, a busca pelo ótimo pode resultar em pontos
críticos locais. Donde, dependendo do problema, tais resultados chegam a ser
satisfatórios. No entanto, quando o objetivo é estritamente encontrar os máximos
ou mínimos globais, o trabalho necessário acaba sendo mais custoso, pelo fato de
que há a incerteza da existência de mais pontos na grande maioria dos problemas.

Ademais, o conjunto $A$, o qual contém os valores aplicáveis no problema, podem
respeitar restrições, fazendo com que o campo de respostas do problema, não seja
contínuo.

\section{{Otimização de Funções à Uma Variável Real}}\label{otim_uma_var}

\hspace{0.8cm}

Evidentemente, as funções possuem as variáveis dependentes (que representam o
objeto da otimização) e as variáveis independentes (que suas grandezas podem
ser selecionadas), podemos denotar que, para a equação
\begin{equation}
    y = f(x),
\end{equation}
quando buscamos otimizá-la, temos como objetivo encontrar valores, \(x\), que
quando aplicados à \(f(x)\), temos o mínimo ou máximo valor \textit{y} (seja
ele local, ou preferencialmente, global).

Partindo dessa perspectiva, acaba surgindo a necessidade de utilizar algum
recurso para encontrar os pontos críticos. E nesse sentindo, pode-se utilizar
a técnica de \textbf{derivação}, que oferece o recurso de identificar tais
pontos.


Podemos observar um fator interessante; por exemplo,
quando a função está num ponto máximo, existem duas possibilidades, a primeira,
a função para de crescer e, em seguida, torna-se indefinida; e a
segunda possibilidade, quando a função para de crescer e começa a decrescer.
Com isso, é importante ressaltar que por definição, quando a derivada (taxa de
variação) em um ponto é positiva, a função cresce, e quando é negativa a função
decresce. Conclui-se que, quando a taxa de variação é zero, a função ou para de
crescer ou de decrescer, o que configura um ponto de máximo ou de mínimo. Mas
também, é quando podo ocorrer o que denominamos ponto de sela (é um
comportamento errático da função, que precisa de uma análise mais refinada). Ou
em última instância, quando a função é constante.

Com o uso da derivada, podemos pensar num método de otimização bastante simples,
considerando \(f(x)\) a função que queremos otimizar e \(f'(x)\) sua função
derivada, podemos dizer que o conjunto $O$, abaixo definido, possui todos os ótimos locais e
globais de \(f(x)\):

\begin{equation}
    O := \{f(x) | f'(x) = 0\}.
    \label{equacao_conjunto_o}
\end{equation}


Portanto, aplicando um filtro em $O$ para obter o máximo e mínimo do conjunto,
acabamos por obter o máximo e mínimo de \(f(x)\):


\begin{equation}
    max(f(x)) = max(O),
\end{equation}

\begin{equation}
    min(f(x)) = min(O).
\end{equation}


Então podemos perceber três problemas:

\begin{itemize}
  \item Determinar como encontrar os pontos onde a derivada se anule, isto é, os pontos críticos;
  \item Determinar se temos de fato todos os pontos;
  \item Classificação dos pontos críticos, que nos revela quais são os máximos e
  os mínimos, e, ademais, da sua relevância quanto a serem locais ou globais.
\end{itemize}

Consideraremos, por agora, apenas o problema de encontrar os pontos críticos.


A taxa de variação de uma função, \(y=f(x)\), em relação a \(x\), é dada pela
relação \(\Delta y / \Delta x\). Sendo o cálculo da derivada
demonstrado pela Figura \ref{derivada_padrao}, o que nos leva ao fato de que
os pontos críticos provém do conjunto solução da equação \(f'(x) = 0\), como
definido na equação \ref{equacao_conjunto_o}. Além disso, caso a função esteja
definida num intervalo fechado, temos pelo Teorema do Valor Absoluto, a
garantia que ela atingirá o valor dos seus pontos extremos, ou seja, seus
valores de máximo e mínimo serão atingidos naquele intervalo. O que culmina
nossa busca pelos pontos críticos.

\begin{figure}[h]
    \centering
    \includegraphics[width=0.45\textwidth]{src/derivada_padrao.jpg}
    \captionsetup{justification=centering}
    \caption{
        Derivada: taxa de variação \(dy/dx\).\\
        \tiny Essa imagem foi coletada de \url{https://brasilescola.uol.com.br}.
    }
    \label{derivada_padrao}
\end{figure}

Partindo desse pressuposto, é nítido que a derivada de uma função é também uma
função. Portanto, fica evidente que uma função pode ser derivada
mais de uma vez, sendo essas derivadas, denominadas de ``primeira derivada'',
``segunda derivada'' e por ai em diante. De modo que, a segunda derivada é
a derivada da primeira derivada. Concluindo-se que dada a função
\(f(x)\), sua primeira derivada é \(df/dx = p(x)\), e sua segunda derivada
\(dp/dx = s(x)\).

Agora vamos tratar da classificação dos pontos críticos.

É sabido que a primeira derivada representa a taxa de variação instantânea de
um ponto na curva, e, que, a segunda derivada proporciona informações
complementares, como por exemplo, se é um ponto de máximo, mínimo ou inflexão,
de modo que, ela determina a concavidade da curva naquele ponto. Como pode ser
visto na Figura \ref{relacao_primeira_segunda_derivada}. E, de modo construtivo,
devemos ressaltar que é importante o estudo do gráfico das derivadas de uma
função, pois é necessariamente através do uso da interpretação do comportamento
da primeira derivada e/ou da segunda derivada, que obtemos a classificação dos
pontos críticos seguindo os critérios que são apontados na Figura \ref{relacao_primeira_segunda_derivada}.

\begin{figure}[h]
    \centering
    \includegraphics[width=0.65\textwidth]{src/relacao_primeira_segunda_derivada.jpg}
    \captionsetup{justification=centering}
    \caption{
        Propriedades e relação da primeira e segunda derivada.\\
        \tiny Essa imagem foi coletada de \url{https://www.alfaconnection.pro.br}.
    }
    \label{relacao_primeira_segunda_derivada}
\end{figure}


\section{{Programando o Método}}

\hspace{0.8cm}


A própria definição da derivada já nos oferece uma visão simples de como
ela pode ser implementada num programa de computador. Mas, com alguns nuances
que devem ser levados em consideração, segue a tal implementação:


\begin{lstlisting}[caption=Calculo de Derivada em Rust.]

pub fn derive1x1_v1<F>(f: &F, x: f64) -> f64
where
    F: Fn(f64) -> f64,
{
    (f(x + DBL_EPS) - f(x)) / DBL_EPS
}


\end{lstlisting}


Essa implementação é ingênua no que diz respeito a precisão da operação. O
uso, depende da aplicação, caso se queira prezar por velocidade de cálculo e
muito pouco sobre precisão, então provavelmente, essa solução seja boa o
suficiente. O problema com a precisão desta implementação é que não se é
levado em consideração o aspecto infinitesimal da variação (\(\Delta x\) na
Figura \ref{derivada_padrao}, ou a variável \textit{h} na implementação
em Rust), já que é por este aspecto que definimos a derivada. Não temos como ter
uma variável com tal propriedade num programa de computador. Como consequência,
\(f(x + h) - f(x)\) já é calculada de forma falha, entregando uma variação
diferente da que seria quando se tem uma variável infinitesimal. O que de fato
é entregue, é a variação um pouco mais à frente do ponto esperado.

A partir daí podemos reformular, considerando esse deslocamento, redefinindo
da seguinte forma:


\begin{equation}
    f'(x) = \frac{f(x + h) - f(x - h)}{2h}.
\end{equation}

Que é equivalente à:

\begin{equation}
    f'(x) = \frac{f(x + h) - f(x - h)}{(x + h) - (x - h)}.
\end{equation}


Usaremos essa segunda formulação por motivos de custo de multiplicação de de
números reais se comparado a somas e subtrações, além da possível perda de
precisão. Assim, conseguimos compensar o tal deslocamento numa nova função
insignificantemente mais custosa, e mais precisa. Como vemos no código a seguir:

\begin{lstlisting}[]

pub fn derive1x1_v2<F>(f: &F, x: f64) -> f64
where
    F: Fn(f64) -> f64,
{
    let x1 = x - h;
    let x2 = x + h;

    let y1 = f(x1);
    let y2 = f(x2);
    return (y2 - y1) / (x2 - x1);
}


\end{lstlisting}



\section{{Otimização de Funções à Várias Variáveis}}

\hspace{0.8cm}
Quando encontramos a necessidade de trabalhar com mais dimensões, que na
otimização se traduz no fato de a função trabalhar com mais parâmetros a serem
otimizados, e a saída da função não necessariamente ser um valor real,
temos a necessidade de trabalharmos com funções que tanto na entrada como na
saída, se utilizam de vetores que podem conter uma ou mais variáveis. Portanto,
é necessário estender as ferramentas criadas para funções à uma variável, as
quais apresentamos na seção \ref{otim_uma_var}, para funções de várias
variáveis. Que formalmente se apresenta da seguinte forma:

% TODO: Colocar nesse modelo as demais Definições já escritas
\begin{definition}[Funções Reais à Várias Variáveis]
    Consideramos uma função real à $n$ variáveis uma função $g$ definida em um
    subconjunto $D$ de \(\mathbb{R}^n\) com imagem em \(\mathbb{R}^m\) para
    $n$, $m$ \(\in \mathbb{N^*}\). Isto é,

    \begin{equation}
            g: D \subseteq \mathbb{R}^n \rightarrow \mathbb{R}^m.
    \end{equation}

    Consideraremos $g(x)$ = $y$, com o $x=(x_1, x_2, ..., x_n)$
    e $y=(y_1, y_2, ..., y_n)$.

\end{definition}



Antes de seguirmos, podemos ajustar um pouco a função para facilitar o
processo de otimização. Como a imagem da função pode ser um vetor com
mais de uma entrada, torna-se um pouco mais difícil decidir qual a melhor
imagem. Sendo assim, pode ser muito conveniente reduzir o vetor em uma
única medida. Uma das formas de fazer isto é pela norma euclidiana.
Portanto, podemos definir uma nova função para ser otimizada a partir da
original, da seguinte forma:


\begin{equation}
        f(x) = ||g(x)|| = ||y||.\\
\end{equation}
        Onde a norma utilizada é dada pela expressão:
\begin{equation}
        ||y|| = \sqrt{ y_1^2 + y_2^2 + \hdots + y_m^2}.
\end{equation}

Já tendo em mãos um bom formato para as funções que desejamos otimizar, agora
precisamos definir como se dá o processo de derivação para tais funções.

% TODO: Provar essa proposição
Para isto, basta estender o processo que já temos para funções à uma variável,
de forma que ainda seja simples calcular. Vamos começar imaginando a seguinte
situação: temos uma função que define um plano, cuja entrada é um vetor de
duas variáveis e que tem como saída um valor real, de acordo com a definição
dada acima. Para que possamos usar o processo definido para funções à uma
variável, devemos considerar apenas uma secção da superfície dessa função.
Assumindo que todos os elementos do vetor de entrada da função são sempre
constantes, exceto o elemento referente à dimensão que queremos derivar,
então obtemos a derivada na dimensão escolhida. Se repetirmos o processo para
a outra variável do vetor de entrada, teremos então dois vetores que
representam as derivadas em cada direção das seções de corte.

Os argumentos que fizemos nos mostram como replicar o processo de obtenção do
que é a derivada pontual de uma função, agora à duas variáveis. A seguir
apresentamos a definição de derivada parcial:

\begin{definition}[Derivadas Parciais]

    Vale salientar que o cálculo do limite para funçãos à duas variáveis
    difenrencia-se no caso de funçãoes reais à uma variável. Quando calculamos
    o limite de funções reais à duas variáveis, devemos atentar para o fato
    de que aqui fazemos uso da aproximação por curvas arbritárias, o que
    dinamiza a complexidade do cálculo de tal limite, fazendo deste um elemento
    muito mais sofisticado do que o que tiamos no caso à uma variável.

    Considerando uma função real $f$ definida em um subconjunto $D$ $\subseteq$ \(\mathbb{R}^2\). Definimos como as derivadas parciais de $f(x, y)$ os seguintes limites, quando existirem:

    \begin{equation}
        \begin{array}{ccc}
            &   \displaystyle {\dfp{f}{x}(x, y) = \lim\limits_{h \to 0} \frac{f(x+h, y) - f(x, y)}{h},}\\
            &\\
            &   \displaystyle {\dfp{f}{y}(x, y) = \lim\limits_{h \to 0} \frac{f(x,y+h) - f(x,y)}{h}.}\\
        \end{array}
    \end{equation}

\end{definition}

Consideramos que a função é diferenciável no ponto \( (x_0, y_0) \) quando as
derivadas paciais existirem para tal ponto. Vale ressaltar que ambos os limites
devem concomitantemente existir. E chamamos essas derivadas como derivada
parcial em x e em y, respectivamente.

Equivalentemente a como fizemos para funções à uma variável podemos agora
definir o que seria a derivada de uma função à duas variáveis.

\begin{definition}[Diferencial de $f(x, y)$]
    Considerando uma função de $f(x, y)$ diferenciável, definimos o diferencial
    de $f(x, y)$ como sendo:

    \begin{equation}
        Df = \dfp{f}{x} \cdot dx + \dfp{f}{y} \cdot dy.
    \end{equation}

\end{definition}

Observe que uma condição \textit{sine qua non} para que a função seja
diferenciável é que exista o diferencial. Pois isto implica na existência das
derivadas paciais que são coeficientes da combinação linear que define o
diferencial.

Estamos agora em posição de definir um instrumento de grande valor para a
análise de otimização de funções reais a duas variáveis, que denominaremos de
gradiente de $f(x, y)$.

\begin{definition}[Gradiente de $f(x, y)$]

    Considerando uma função de $f(x, y)$ diferenciável e o diferencial como
    definido acima, o vetor gradiente de $f(x, y)$ é aquele que satisfaz:

    \begin{equation}
        Df \cdot v = \left\langle \nabla f, v \right\rangle.
    \end{equation}

    Vemos que o $\nabla f(x, y)$ = $\left(\dfp{f}{x}, \dfp{f}{y}\right)$.


\end{definition}

    %TODO: Mostrar uma proposição (mostrar que o gradiente é a direção em que a função cesce ou decresce mais rapidamente [usar cos talvez])
    \begin{proposition}
        TODO: $\nabla$ é a direção de otimalidade de função
    \end{proposition}

% TODO: Definir pontos críticos
% TODO: Definir curvas de nível
% TODO: Definir matriz jacobiana (Falar da J nula)
% TODO: Definir matriz Hessiana (Falar da H nula)
% TODO: Classificar pontos críticos a partir da Hessiana
% TODO: Ver teorema espectral
% Pontos criticos = entradas do gradiente nulas; jacobiana nula=pontos criticos
% Dica: uso do polinomio de taylor para duas variaveis

% TODO: Isso aqui está muito resumido, aprimorar isso
Podemos considerar esse mesmo processo para funções cujos vetores de entrada são
maiores, já que teremos apenas um gradiente com mais derivadas parciais, e uma
vez tendo em mãos este vetor, basta modificar as entradas até que todos os
elementos se anulem, como era feito com funções à uma variável visto na
seção \ref{otim_uma_var}, sendo esses valores de entrada, o resultado do
processo de otimização.




%
